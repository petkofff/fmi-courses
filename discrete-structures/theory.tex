\documentclass[12pt]{article}

    \usepackage{amsmath}
    \usepackage{amsfonts}
    \usepackage[utf8]{inputenc}
    \usepackage[bulgarian]{babel}


    \title{Дискретни структури, Теория}
    \author{Георги Петков Петков}
    \date{}

    \begin{document}
    \maketitle
    \section{Логика}
    \subsection*{Свойства на логическите съюзи}
    Свойство на константите:
    \begin{gather*}
        p \land T \equiv p \\
        p \lor T \equiv T \\
        p \land F \equiv F \\
        p \lor F \equiv p \\
    \end{gather*}
    Закон на двойното отрицавие: \( \neg(\neg p) \equiv p \) \\ \\
    Комутативност: 
    \begin{gather*}
        p \land q \equiv q \land p \\
        p \lor q \equiv q \lor p \\
        p \leftrightarrow q \equiv q \leftrightarrow p \\
        p \oplus q \equiv q \oplus p \\
    \end{gather*}
    Асоциативност:
    \begin{gather*}
        (p \land q) \land r \equiv p \land (q \land r) \equiv p \land q \land r \\
        (p \lor q) \lor r \equiv p \lor (q \lor r) \equiv p \lor q \lor r \\
    \end{gather*}
    Дистрибутивност:
    \begin{gather*}
        p \land (q \lor r) \equiv (p \land q) \lor (p \land r) \\
        p \lor (q \land r) \equiv (p \lor q) \land (p \lor r) \\ 
    \end{gather*}
    Де Морган:
    \begin{gather*}
        \neg (p \land q) \equiv \neg p \lor \neg q \\
        \neg (p \lor q) \equiv \neg p \land \neg q \\
    \end{gather*}
    Свойство на импликацията: \( p \to q \equiv \neg p \lor q \) \\ \\
    Свойство на би-импликацията: \( p \leftrightarrow q \equiv ( p \to q ) \land ( q \to p ) \) \\ \\
    Иденпотентност:
    \begin{gather*}
        p \land p \equiv p \\
        p \lor p \equiv p 
    \end{gather*}

    \subsection*{Предикатна логика}
    \( n \in \mathbb{N}, A = \{ a_1, a_2, \dots a_n \} \) - домейн, универсален квантор (\( \forall \)), екзистенциален квантор (\( \exists \)) \\
    \begin{gather*}
        \forall x P(x) \equiv \land^{i=1}_{n} P(a_i) \\
        \exists x P(x) \equiv \lor^{i=1}_{n} P(a_i) \\
        \neg \forall x P(x) \equiv \exists x \neg P(x) \\
        \neg \exists x P(x) \equiv \forall x \neg P(x)
    \end{gather*}
    \section{Множества}
    \subsection*{Аксиоми:}
    Аксиома за обема: $ \forall a (a \in A \iff a \in B) \implies A=B $ \\
    Аксиома за определянето: $ M'=\{ x \in M, \pi (x) \} $ \\
    Аксиома за степенното множество: $ 2^A $ е множеството от всички подмножества на множеството $A$.
    \end{document}
